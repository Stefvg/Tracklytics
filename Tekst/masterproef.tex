\documentclass[master=cws,masteroption=se]{kulemt}
\setup{title={DevOps: Operational data for developers},
  author={Stef Van Gils},
  promotor={Prof.\ W. Joosen},
  assessor={Dimitri Van Landuyt},
  assistant={Dimitri Van Landuyt}}
% De volgende \setup mag verwijderd worden als geen fiche gewenst is.
\setup{filingcard,
  translatedtitle={DevOps: Operational data for developers},
  udc=621.3,
  shortabstract={}}
% Verwijder de "%" op de volgende lijn als je de kaft wil afdrukken
%\setup{coverpageonly}
% Verwijder de "%" op de volgende lijn als je enkel de eerste pagina's wil
% afdrukken en de rest bv. via Word aanmaken.
%\setup{frontpagesonly}

% Kies de fonts voor de gewone tekst, bv. Latin Modern
\setup{font=lm}

% Hier kun je dan nog andere pakketten laden of eigen definities voorzien

% Tenslotte wordt hyperref gebruikt voor pdf bestanden.
% Dit mag verwijderd worden voor de af te drukken versie.
\usepackage[pdfusetitle,colorlinks,plainpages=false]{hyperref}

%%%%%%%
% Om wat tekst te genereren wordt hier het lipsum pakket gebruikt.
% Bij een echte masterproef heb je dit natuurlijk nooit nodig!
\IfFileExists{lipsum.sty}%
 {\usepackage{lipsum}\setlipsumdefault{11-13}}%
 {\newcommand{\lipsum}[1][11-13]{\par Hier komt wat tekst: lipsum ##1.\par}}
%%%%%%%

%%%%%%%
% Gebruikt om \texttt{} line breaks uit te voeren indien nodig
\newcommand*\justify{%
  \fontdimen2\font=0.4em% interword space
  \fontdimen3\font=0.2em% interword stretch
  \fontdimen4\font=0.1em% interword shrink
  \fontdimen7\font=0.1em% extra space
  \hyphenchar\font=`\-% allowing hyphenation
}
%%%%%%%

%\includeonly{hfdst-n}
\begin{document}

\begin{preface}
  Dit is mijn dankwoord om iedereen te danken die mij bezig gehouden heeft.
  Hierbij dank ik mijn promotor, mijn begeleider en de voltallige jury.
  Ook mijn familie heeft mij erg gesteund natuurlijk.
\end{preface}

\tableofcontents*

\begin{abstract}
  In dit \texttt{abstract} environment wordt een al dan niet uitgebreide
  samenvatting van het werk gegeven. De bedoeling is wel dat dit tot
  1~bladzijde beperkt blijft.

  \lipsum[1]
\end{abstract}

% Een lijst van figuren en tabellen is optioneel
%\listoffigures
%\listoftables
% Bij een beperkt aantal figuren en tabellen gebruik je liever het volgende:
\listoffiguresandtables
% De lijst van symbolen is eveneens optioneel.
% Deze lijst moet wel manueel aangemaakt worden, bv. als volgt:

% Nu begint de eigenlijke tekst
\mainmatter

%\chapter{Inleiding}
\label{inleiding}
In dit hoofdstuk wordt het werk ingeleid. Het doel wordt gedefinieerd en er
wordt uitgelegd wat de te volgen weg is (beter bekend als de rode draad).

Als je niet goed weet wat een masterproef is, kan je altijd
Wikipedia\cite{wiki} eens nakijken.

\section{Lorem ipsum 4--5}
\lipsum[4-5]

\section{Lorem ipsum 6--7}
\lipsum[6-7]

%%% Local Variables: 
%%% mode: latex
%%% TeX-master: "masterproef"
%%% End: 

%\chapter{Het eerste hoofdstuk}
\label{hoofdstuk:1}
Een hoofdstuk behandelt een samenhangend geheel dat min of meer op zichzelf
staat. Het is dan ook logisch dat het begint met een inleiding, namelijk
het gedeelte van de tekst dat je nu aan het lezen bent.

\section{Eerste onderwerp in dit hoofdstuk}
De inleidende informatie van dit onderwerp.

\subsection{Een item}
De bijbehorende tekst. Denk eraan om de paragrafen lang genoeg te maken en
de zinnen niet te lang.

Een paragraaf omvat een gedachtengang en bevat dus steeds een paar zinnen.
Een paragraaf die maar \'e\'en lijn lang is, is dus uit den boze.

\section{Tweede onderwerp in dit hoofdstuk}
Er zijn in een hoofdstuk verschillende onderwerpen. We zullen nu
veronderstellen dat dit het laatste onderwerp is.

\subsection{Een item}
Maak ook geen misbruik van opsommingen. Voor korte opsommingen gebruik je
geen ``\verb|itemize|'' of ``\texttt{enumerate}'' commando's. Doe dus
\emph{niet} het volgende:
\begin{quote}
  De Eiffeltoren bevat drie verdiepingen:
  \begin{itemize}
  \item de eerste;
  \item de tweede;
  \item de derde.
  \end{itemize}
\end{quote}
Maar doe:
\begin{quote}
  De Eiffeltoren bevat drie verdiepingen: de eerste, de tweede en de derde.
\end{quote}

\section{Besluit van dit hoofdstuk}
Als je in dit hoofdstuk tot belangrijke resultaten of besluiten gekomen
bent, dan is het ook logisch om het hoofdstuk af te ronden met een
overzicht ervan. Voor hoofdstukken zoals de inleiding en het
literatuuroverzicht is dit niet strikt nodig.

%%% Local Variables: 
%%% mode: latex
%%% TeX-master: "masterproef"
%%% End: 

\include{architectuur}
\include{implementatie}
% ... en zo verder tot
%\chapter{Het laatste hoofdstuk}
\label{hoofdstuk:n}
Een hoofdstuk behandelt een samenhangend geheel dat min of meer op zichzelf
staat. Het is dan ook logisch dat het begint met een inleiding, namelijk
het gedeelte van de tekst dat je nu aan het lezen bent.

\section{Eerste onderwerp in dit hoofdstuk}
De inleidende informatie van dit onderwerp.

\subsection{Een item}
De bijbehorende tekst. Denk eraan om de paragrafen lang genoeg te maken en
de zinnen niet te lang.

Een paragraaf omvat een gedachtengang en bevat dus steeds een paar zinnen.
Een paragraaf die maar \'e\'en lijn lang is, is dus uit den boze.

\section{Tweede onderwerp in dit hoofdstuk}
Er zijn in een hoofdstuk verschillende onderwerpen. We zullen nu
veronderstellen dat dit het laatste onderwerp is.

\section{Besluit van dit hoofdstuk}
Als je in dit hoofdstuk tot belangrijke resultaten of besluiten gekomen
bent, dan is het ook logisch om het hoofdstuk af te ronden met een
overzicht ervan. Voor hoofdstukken zoals de inleiding en het
literatuuroverzicht is dit niet strikt nodig.

%%% Local Variables: 
%%% mode: latex
%%% TeX-master: "masterproef"
%%% End: 

%\chapter{Besluit}
\label{besluit}
De masterproeftekst wordt afgesloten met een hoofdstuk waarin alle
besluiten nog eens samengevat worden. Dit is ook de plaats voor suggesties
naar het verder gebruik van de resultaten, zowel industri"ele toepassingen
als verder onderzoek.

\lipsum[1-7]

%%% Local Variables: 
%%% mode: latex
%%% TeX-master: "masterproef"
%%% End: 


% Indien er bijlagen zijn:
%\appendixpage*          % indien gewenst
%\appendix
%\chapter{De eerste bijlage}
\label{app:A}
In de bijlagen vindt men de data terug die nuttig kunnen zijn voor de
lezer, maar die niet essentieel zijn om het betoog in de normale tekst te
kunnen volgen. Voorbeelden hiervan zijn bronbestanden,
configuratie-informatie, langdradige wiskundige afleidingen, enz.

In een bijlage kunnen natuurlijk ook verdere onderverdelingen voorkomen,
evenals figuren en referenties\cite{h2g2}.

\section{Meer lorem}
\lipsum[50]

\subsection{Lorem 15--17}
\lipsum[15-17]

\subsection{Lorem 18--19}
\lipsum[18-19]

\section{Lorem 51}
\lipsum[51]

%%% Local Variables: 
%%% mode: latex
%%% TeX-master: "masterproef"
%%% End: 

% ... en zo verder tot
%\chapter{De laatste bijlage}
\label{app:n}
In de bijlagen vindt men de data terug die nuttig kunnen zijn voor de
lezer, maar die niet essentieel zijn om het betoog in de normale tekst te
kunnen volgen. Voorbeelden hiervan zijn bronbestanden,
configuratie-informatie, langdradige wiskundige afleidingen, enz.

\section{Lorem 20-24}
\lipsum[20-24]

\section{Lorem 25-27}
\lipsum[25-27]

%%% Local Variables: 
%%% mode: latex
%%% TeX-master: "masterproef"
%%% End: 


\backmatter
% Na de bijlagen plaatst men nog de bibliografie.
% Je kan de  standaard "abbrv" bibliografiestijl vervangen door een andere.
\bibliographystyle{abbrv}
\bibliography{referenties}



\end{document}

%%% Local Variables: 
%%% mode: latex
%%% TeX-master: t
%%% End: 
